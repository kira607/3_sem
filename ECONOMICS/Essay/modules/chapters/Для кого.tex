\section{Для кого?}

Чтобы определить эффективность режима самозанятости, разберем в первую очередь, для кого он был создан.

Переходить на самозанятость могут все физические лица РФ. 
При этом для перехода на самозанятость не обязательно 
отказываться от трудового договора,
если человек уже работает с официальным работодателем.
Гражданин РФ имеют право одновременного работать по трудовому договору 
и получать дополнительный доход будучи самозанятым. 

Например, некий гражданин А. работает в
ГБОУ СОШ № 31 Василеостровского района
учителем старших классов, а в свободное
подрабатывает репетитором для частных учеников. 
Чтобы получать профессиональных доход от репетиторской деятельности
в рамках закона, его необходимо задекларировать как доход,
получаемый в результате деятельности,
осуществляемой в рамках самозанятости.

Самозанятость подразумевает не только получение дополнительного дохода от, 
например, репетиторства, 
но также и получение основного дохода от таких деятельностей, как \cite{kontur}:

\begin{itemize}
	\item оказание услуг на дому, например, косметических;
	\item фото- и видеосъемка на заказ;
	\item реализация продукции собственного производства;
	\item проведение мероприятий и праздников;
	\item юридические консультации и ведение бухгалтерии;
	\item удаленная работа через электронные площадки, например, через биржи фриланса (fl.ru, workzilla);
	\item сдача квартиры в аренду посуточно или на долгий срок;
	\item услуги по перевозке пассажиров и грузов;
	\item строительные работы и ремонт помещений.
\end{itemize}

Как же отличить эту грань между самозанятым и ИП? 
Ведь индивидуальные предприниматели точно также оказывают услуги, 
реализуют товары, мероприятия и другое.

Есть несколько отличительных пунктов: 

\begin{enumerate}
	\item Cамозанятые не имеют право нанимать сотрудников, так как не смогут составить действующий договор, таким образом, самозанятые получают доход только от личного труда 
	
	\item Максимальный доход в сумме за 12 месяцев с момента оформления (дата подачи документов в налоговый орган) и далее не может составлять более \textbf{2,4} млн рублей \cite{kontur}.
\end{enumerate}

Важно различать доход от профессиональной деятельности и разовый доход.
Например, гражданка Б. любит и отлично умеет плести косы. 
Её подруга попросила своей дочке на выпускной сделать прическу из кос. 
Гражданка Б. взялась за работу с целью помочь своей подруге, 
а та внезапно предложила ей хотя бы 300 рублей на оплату материалов. 
В этом случае, гражданка Б. не обязана сообщать налоговой о неожиданном доходе, 
так как этот доход был разовым и был своего рода наградой за помощь.
Однако, если гражданка Б. решит на постоянной основе делать 
прически девушкам на выпускной и брать за это оплату (даже на материалы), 
это подлежит статье получения профессионального дохода. 

Получается, что практически каждый человек в стране может стать самозанятым.
Главное 
получать до \textbf{2,4} млн рублей в год, 
занимаясь собственным делом
без привлечения других работников,
и вовремя выплачивая налог.
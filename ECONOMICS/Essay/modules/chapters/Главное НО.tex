\section{Главное НО}

Каждый из нас хотел бы получать поддержку
от государства в старости. 
Пенсия является важным инструментом поддержки 
нетрудоспособных слоёв населения.

В России достаточно спорная ситуация с пенсиями.
Мой дедушка всю жизнь работал, не покладая рук 
и обеспечивая семью. Сейчас его пенсия составляет 
не более 20 тыс. рублей, что несколько выше средней пенсии 
в России (16 000 рублей в месяц \cite{mid-pens}). 
Этих денег в среднем хватает только еду
и вещи первой необходимости, и речь уже не заходит 
о новой одежде или бытовой технике.

Самозанятый мог бы подумать, что он приносит пользу обществу,
платит налоги, является важным для развития общества и государства
и может рассчитывать на достойную пенсию, но $\ldots$

Стаж работы в рамках самозанятости не учитывается при расчёте пенсии.
То есть если человек параллельно официально не трудоустроен, то
максимум он может рассчитывать на социальную пенсию. 
На момент написания эссе сумма социально пенсии составляет $ 5283,84 $ руб. \cite{pens-quote}

\begin{displayquote}
	\textit{
		В 2020 году женщины смогут получать социальную пенсию по старости с 61,5 лет, а мужчины — с 66,5 лет. В 2028 году возраст назначения социальной пенсии составит 65 и 70 лет соответственно.
	}
\end{displayquote}

Сможет ли человек после 60-ти лет заниматься тем делом,
которое он выбрал в качестве деятельности в рамках самозанятости?
Зависит от деятельности, но в этом случае на пенсию рассчитывать не стоит.
\begin{comment}
	Самозанятость в РФ, условия, организация, эффективность.
	https://kontur.ru/articles/4818
	для презы: https://mbk.kmscity.ru/assets/news/2020/samozaniatye.pdf
	https://vld-invest.ru/images/---.pdf
	https://www.yarregion.ru/depts/der/SiteAssets/Pages/%D0%9F%D1%80%D0%B5%D0%B4%D0%BF%D1%80%D0%B8%D0%BD%D0%B8%D0%BC%D0%B0%D1%82%D0%B5%D0%BB%D1%8C%D1%81%D1%82%D0%B2%D0%BE/statyi/%D0%9D%D0%B0%D0%BB%D0%BE%D0%B3%20%D0%BD%D0%B0%20%D0%BF%D1%80%D0%BE%D1%84%D0%B5%D1%81%D1%81%D0%B8%D0%BE%D0%BD%D0%B0%D0%BB%D1%8C%D0%BD%D1%8B%D0%B9%20%D0%B4%D0%BE%D1%85%D0%BE%D0%B4.pdf
\end{comment}

\begin{comment}
	Основное из методички по эссе, что нужно проверить в моём текст и отредактировать его, согласно нижесказанному:
	
	1) Эссе НЕ может содержать много тем или идей (мыслей). 
	Оно отражает и развивает только один вариант, одну мысль. 
	Эссе - есть ответ на этот един-ственный вопрос.
	
	= у нас основной вопрос заключается в эффективности режима самозанятости
	
	2) Введение: 
	суть и обоснование выбора данной темы эссе. 
	Обязательным является изложение причин написания эссе. 
	Почему эта тема интересна автору и должна также быть интересна читателю? 
	Во введении необходимо обосновать актуальность выбранной темы, 
	сформулировать цель и задачи исследования, 
	а также дать краткое определение используемых в работе понятий и ключевых терминов. 
	Однако их количество в эссе не должно быть излишне большим (как правило, три или четыре).
	
	3) Развитие темы: 
	аргументированное раскрытие темы на основе собранного материала (идеи, модели и данные). 
	Данная часть работы предполагает развитие авторской аргументации и анализа исследуемой проблемы, 
	а также обоснование выводов на основе имеющихся данных, 
	положений экономиче-ской теории и фактологического материла.
	
	4) Заключение: 
	обобщение материала и аргументированные выводы по теме с указанием 
	возможных путей решения исследуемой проблемы в форме крат-кого изложения основных аргументов автора. 
	При этом следует помнить, что заключение должно быть очень кратким (0,5 – 1,0 страница).
	
	5) Наиболее часто встречающиеся ошибки при написании эссе:
	
	Неумение придерживаться ответа на основной вопрос, плохая органи-зация ответа.
	
	Использование риторики (утверждение) вместо аргументации (доказательства).
	
	Небрежное оперирование данными, включая чрезмерное обобщение.
	
	Слишком обширная описательная часть, не подкрепленная аналитическим материалом.
	
	Изложение других точек зрения, без высказывания собственной позиции.
	
	6) Ориентация текста книжная (лист располагается вертикально).
	Текст наносится постранично только с одной стороны листа, двустороннее располо-жение текста на листе недопустимо.
	
	Поля и отступы текста: 
	левое поле – 30 мм, 
	правое поле – 15 мм, 
	верхнее и нижнее поля – по 25 мм. 
	Если текст печатается на компьютере в текстовом редакторе Microsoft Word, 
	то при форматировании следует установить правый ограничитель текста в позицию 16 ½
	
	7) Нумерация страниц начинается с титульного листа. 
	Титульный лист считается первой страницей, но номер «1» на нем не проставляется. 
	На второй странице располагается «План» работы. 
	Такие разделы работы как 
	«Введение», «Заключение», «Литература» и «Приложения» приводятся в «Плане», но не нумеруются. 
	Нумерации подлежат только разделы, относящиеся к основной части работы.
	
	На титульном листе должна содержаться следующая информация: 
	наименование вуза, 
	кафедра, по которой выполняется работа, 
	название темы, 
	аббревиатура студенческой группы, 
	фамилия и инициалы студента, 
	фамилия и инициалы научного руководителя, 
	а также его ученая степень и должность, город и текущий год.
	
	8) Источники в списке использованной литературы позиционируются следующим образом:
	
	1. Нормативно-правовые акты:
	
	а) законы;
	б) указы Президента РФ;
	в) законодательные акты Федерального собрания РФ;
	г) постановления Правительства РФ;
	д) письма, инструкции, распоряжения Министерств и ведомств РФ;
	
	2. Книги (монографии, сборники);
	
	3. Периодические издания;
	4. Статистические сборники и справочники;
	5. Печатные материалы на иностранных языках;
	6. Интернет-ресурсы.
	
	Для всех литературных источников списка сохраняется непрерывная сквозная нумерация.
\end{comment}

\section*{Введение}
\addcontentsline{toc}{section}{Введение}

Мир не стоит на месте. 

Привычный уклад жизни, приемлемый 30 лет назад,
сегодня кажется невозможным. 
Технологии связали если не все, то большую часть уголков всего мира.
Благодаря этому личное время почти каждого человека стало 
недорогостоящим и доступным ресурсом. 
Теперь уже не приходится растрачивать время 
на покупки, так как есть доставка,
на поиск информации, так как есть интернет,
на чтение книг, так как есть аудиокниги,
и на многое другое.

Благодаря постепенному развитию всех сфер жизни
каждый человек способен заниматься любимым делом,
отдавать ему своё свободное время. 
Рано или поздно у некоторых людей возникает вопрос ---
как монетизировать свои умения?
Как заниматься тем, чем я хочу и получать доход?
Как зарабатывать дистанционно, без привязки к офису? 

Люди, которые задаются этим вопросом, 
ищут материалы по удаленным профессиям,
подбирая их под свои умения,
выбирают подходящую, 
развиваются и успешно работают на себя в “хоум-офисе”,
будь то дизайнер, программист, верстальщик, копирайтер и т.д. 

Делают ли они это легально? 
Что значит "работать на себя"\ без регистрации ИП? 
Нужно ли платить налоги государству, 
беря новый заказ на бирже фриланса или обучая иностранному языку? 

На все эти вопросы отвечает термин самозанятость \cite{self-employment}.

Поговорим о самозанятости в РФ, 
о её условиях, 
причинах появления и ответим на главный вопрос, --- 
насколько система самозанятости эффективна в наше время, 
а самое главное --- для кого.

Исследуем понятие самозанятость в РФ через определение её условий организации 
с целью доказать или опровергнуть эффективность данного вида занятости. 

Дадим определение термина самозанятость,
выявим профессии самозанятых и условия налогообложения, 
перечислиим преимущества и недостатки режима для физических лиц.

На данный момент существует следующее определение самозанятости
\cite{self-employment-wiki}:

\begin{displayquote}
	\textit{
		Самозанятость --- форма получения вознаграждения за свой труд непосредственно от заказчиков, в отличие от наёмной работы.
	}
\end{displayquote}
\section*{Форматы входных и выходных файлов}

\subsection*{Входной файл}

Входной файл должен быть представлен в виде текстового файла, содержащего последовательность чисел.
Разширение файла может быть любым.
Последовательность должна быть записана в следующем виде:
Первым членом последовательности должна быть ширина матрицы $ w $ -- количество столбцов,
далее должна идти высота матрицы $ h $ -- количество строк матрицы.
После должны идти элементы матрицы.
Их количество определяется по формуле $ w*h $.
Между элементами могут быть любые разделительные символы
(пробел, знак переноса строки, знак табуляции).
Рекоментуется оформлять элементы
матрицы в виде сетки $ w*h $ для повышения читаемости файла.

\subsection*{Выходной файл}

Выходной файл записывается в следующем формате:

\begin{itemize}
	\item Матрица 1
	\item Знак '+'
	\item Матрица 2
	\item Знак '='
	\item Матрица 3 (сумма матриц 1 и 2)
\end{itemize}
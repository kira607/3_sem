\section*{Формулировка задания}

Разработать программу, осуществляющую операции создания,
сохранения и восстановления линейного однонаправленного списка по
своему заданию.
Каждый элемент списка должен представлять собой структуру,
состоящую из полей, заданных в своём варианте, а также указателей на
соседний (следующий) элемент списка.
Список должен быть создан в динамической памяти программы с
помощью указателя. Доступ к отдельным элементам списка также должен
осуществляться через указатели.
В программе должно быть предусмотрено меню для выбора тех или
иных операций над списком. Каждая операция над списком должна быть
выполнена в отдельной пользовательской функции.
В программе должны быть предусмотрены следующие операции над
списком:
\begin{itemize}
    \item создание нового списка;
    \item ввод данных списка пользователем с клавиатуры;
    \item сохранение данных списка во внешнем файле;
    \item чтение данных списка (его восстановление) из внешнего файла.
\end{itemize}

\textbf{Индивидуальный вариант №10:}
Список грузовиков транспортной компании (марка, грузоподъёмность,
максимальная дальность перевозки и
т.п.);

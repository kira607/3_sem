\section*{Исследуемые закономерности}

Если вдоль пластины полупроводника, помещенной в магнитное поле,
перпендикулярное вектору плотности тока, а, следовательно, и средней скорости направленного движения
заряженных частиц, то на заряженную частицу, движущуюся со средней скоростью $ \langle v \rangle $, будет действовать
сила Лоренца в направлении, перпендикулярном вектору плотности тока:

\begin{equation}\label{F}
    \overrightarrow{F} = 
    e\langle \overrightarrow{v} \rangle \times \overrightarrow{B}
\end{equation}

В результате между гранями с электродами 1 и 2
появится поперечное электрическое поле. (\textbf{\underline{эффект Холла}}):

\begin{equation}\label{E}
    E_x = \dfrac{F}{e} = \langle v \rangle B
\end{equation}

Сила тока $ I $:

\begin{equation}\label{I}
    I = nbe\langle v \rangle d
\end{equation}

Решая совместно (\ref{F}), (\ref{E}), (\ref{I}) получим:

\begin{equation}\label{U}
    U_x = \dfrac{1}{ne}\times\dfrac{IB}{d} = R\dfrac{IB}{d} 
\end{equation}

где $ R = \dfrac{1}{ne}$ --- постоянная Холла.

Магнитное поле ЭМ в центре симметрии определяется по формуле 

\begin{equation}\label{B}
    B = B_H + a \times I_2
\end{equation}

где, 
$ B_H $ --- начальная индукция магнитного поля сердечника электромагнита, 
$ I_2 $ --- сила тока (А), 
$ a $ – коэффициент пропорциональности 
в диапазоне изменения силы тока в электромагните от 0,1 А до 1 А.
